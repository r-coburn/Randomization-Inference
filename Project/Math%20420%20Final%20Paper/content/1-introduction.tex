%----------------------------------------------------------------------------------------
% Introduction
%----------------------------------------------------------------------------------------
\setcounter{page}{1} % Sets counter of page to 1

\section{Introduction} % Add a section title

In mathematical statistics, a problem that often comes up with basic tests (t-tests, Analysis of Variance, etc) is dealing with small sample sizes. When looking specifically at the two-sample t-test, the condition that has to be met and is taught in all introductory statistics courses is normality of both samples. However, in the case where this conditions cannot be met, we hope to have a sample size greater than $30$. But this magic number $30$ can be misleading. Real world data often contains more than $30$ observations. Creating models such as linear regression models is difficult because real world data often fails to meet critical model conditions such as normality of residuals and homoescedasticity. Because of this, we cannot blindly trust parameters like a $\hat{\beta_i}\text{'s}$ in linear regression models and residual deviance in logistic regression models. There are also times when real world data is too small and then we can’t even use the Central Limit Theorem to assume our statistics come from an approximation of a normal distribution. However, there is a solution to both of these problems: randomization based inference. Another common solution is bootstrapping, however, that is most useful when we want to develop robust estimates for statistics and standard errors as well as counteract sampling error. We are focusing on randomization methods, which focus on randomization of units within our sample. Forgetting all model distribution assumptions, randomization-based inference only cares whether the sample that you have is typical of the population. In this paper, we will explain what randomization based inference is in detail while walking through a short example pointing out each step in the context of our paper, explain why this process works, elaborate on some of the limitations that come with this test, and explore Monte Carlo methods for re-randomization. 
