%----------------------------------------------------------------------------------------
% Conclusion
%----------------------------------------------------------------------------------------

\section{Conclusion}

Throughout this paper, we've discussed the importance of randomization-based inference in statistics, the situations that this method is best attuned to, and most importantly, why this method works for so many statistical hypothesis tests. We've also briefly touched on Monte Carlo Simulations and the importance of these methods in performing accurate randomization-based inference. Randomization-based inference is a powerful non-parametric tool which specifies that we only check one condition: randomness. This creates an extremely versatile process which can be implemented in situations where other model conditions cannot be met in some cases. As with many topics in statistics, we learned that the utility for randomization-based inference derived from the Central Limit Theorem. 

{\color{white}\cite{paradoxPaper}\cite{bayesianAnalysis}\cite{hesterbergTextbook}\cite{howDoPermTestsWork}\cite{monteCarloIntro}\cite{randomizationProcedures}\cite{simpleMonteCarlo}\cite{whatTeachersShouldKnow}\cite{wildBootstrap}}

